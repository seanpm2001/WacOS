\section{\module{CGIHTTPServer} ---
         CGI-capable HTTP request handler}


\declaremodule{standard}{CGIHTTPServer}
  \platform{Unix}
\sectionauthor{Moshe Zadka}{mzadka@geocities.com}
\modulesynopsis{This module provides a request handler for HTTP servers
                which can run CGI scripts.}


The \module{CGIHTTPServer} module defines a request-handler class,
interface compatible with
\class{BaseHTTPServer.BaseHTTPRequestHandler} and inherits behavior
from \class{SimpleHTTPServer.SimpleHTTPRequestHandler} but can also
run CGI scripts.

\strong{Note:}  This module is \UNIX{} dependent since it creates the
CGI process using \function{os.fork()} and \function{os.exec()}.

The \module{CGIHTTPServer} module defines the following class:

\begin{classdesc}{CGIHTTPRequestHandler}{request, client_address, server}
This class is used to serve either files or output of CGI scripts from 
the current directory and below. Note that mapping HTTP hierarchic
structure to local directory structure is exactly as in
\class{SimpleHTTPServer.SimpleHTTPRequestHandler}.

The class will however, run the CGI script, instead of serving it as a
file, if it guesses it to be a CGI script. Only directory-based CGI
are used --- the other common server configuration is to treat special
extensions as denoting CGI scripts.

The \function{do_GET()} and \function{do_HEAD()} functions are
modified to run CGI scripts and serve the output, instead of serving
files, if the request leads to somewhere below the
\code{cgi_directories} path.
\end{classdesc}

The \class{CGIHTTPRequestHandler} defines the following data member:

\begin{memberdesc}{cgi_directories}
This defaults to \code{['/cgi-bin', '/htbin']} and describes
directories to treat as containing CGI scripts.
\end{memberdesc}

The \class{CGIHTTPRequestHandler} defines the following methods:

\begin{methoddesc}{do_POST}{}
This method serves the \code{'POST'} request type, only allowed for
CGI scripts.  Error 501, "Can only POST to CGI scripts", is output
when trying to POST to a non-CGI url.
\end{methoddesc}

Note that CGI scripts will be run with UID of user nobody, for security
reasons. Problems with the CGI script will be translated to error 403.

For example usage, see the implementation of the \function{test()}
function.


\begin{seealso}
  \seemodule{BaseHTTPServer}{Base class implementation for Web server
                             and request handler.}
\end{seealso}
