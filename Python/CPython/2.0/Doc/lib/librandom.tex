\section{\module{random} ---
         Generate pseudo-random numbers}

\declaremodule{standard}{random}
\modulesynopsis{Generate pseudo-random numbers with various common
                distributions.}


This module implements pseudo-random number generators for various
distributions: on the real line, there are functions to compute normal
or Gaussian, lognormal, negative exponential, gamma, and beta
distributions.  For generating distribution of angles, the circular
uniform and von Mises distributions are available.


The \module{random} module supports the \emph{Random Number
Generator} interface, described in section \ref{rng-objects}.  This
interface of the module, as well as the distribution-specific
functions described below, all use the pseudo-random generator
provided by the \refmodule{whrandom} module.


The following functions are defined to support specific distributions,
and all return real values.  Function parameters are named after the
corresponding variables in the distribution's equation, as used in
common mathematical practice; most of these equations can be found in
any statistics text.  These are expected to become part of the Random
Number Generator interface in a future release.

\begin{funcdesc}{betavariate}{alpha, beta}
Beta distribution.  Conditions on the parameters are
\code{\var{alpha} > -1} and \code{\var{beta} > -1}.
Returned values range between 0 and 1.
\end{funcdesc}

\begin{funcdesc}{cunifvariate}{mean, arc}
Circular uniform distribution.  \var{mean} is the mean angle, and
\var{arc} is the range of the distribution, centered around the mean
angle.  Both values must be expressed in radians, and can range
between 0 and \emph{pi}.  Returned values will range between
\code{\var{mean} - \var{arc}/2} and \code{\var{mean} + \var{arc}/2}.
\end{funcdesc}

\begin{funcdesc}{expovariate}{lambd}
Exponential distribution.  \var{lambd} is 1.0 divided by the desired
mean.  (The parameter would be called ``lambda'', but that is a
reserved word in Python.)  Returned values will range from 0 to
positive infinity.
\end{funcdesc}

\begin{funcdesc}{gamma}{alpha, beta}
Gamma distribution.  (\emph{Not} the gamma function!)  Conditions on
the parameters are \code{\var{alpha} > -1} and \code{\var{beta} > 0}.
\end{funcdesc}

\begin{funcdesc}{gauss}{mu, sigma}
Gaussian distribution.  \var{mu} is the mean, and \var{sigma} is the
standard deviation.  This is slightly faster than the
\function{normalvariate()} function defined below.
\end{funcdesc}

\begin{funcdesc}{lognormvariate}{mu, sigma}
Log normal distribution.  If you take the natural logarithm of this
distribution, you'll get a normal distribution with mean \var{mu} and
standard deviation \var{sigma}.  \var{mu} can have any value, and
\var{sigma} must be greater than zero.  
\end{funcdesc}

\begin{funcdesc}{normalvariate}{mu, sigma}
Normal distribution.  \var{mu} is the mean, and \var{sigma} is the
standard deviation.
\end{funcdesc}

\begin{funcdesc}{vonmisesvariate}{mu, kappa}
\var{mu} is the mean angle, expressed in radians between 0 and 2*\emph{pi},
and \var{kappa} is the concentration parameter, which must be greater
than or equal to zero.  If \var{kappa} is equal to zero, this
distribution reduces to a uniform random angle over the range 0 to
2*\emph{pi}.
\end{funcdesc}

\begin{funcdesc}{paretovariate}{alpha}
Pareto distribution.  \var{alpha} is the shape parameter.
\end{funcdesc}

\begin{funcdesc}{weibullvariate}{alpha, beta}
Weibull distribution.  \var{alpha} is the scale parameter and
\var{beta} is the shape parameter.
\end{funcdesc}

\begin{seealso}
  \seemodule{whrandom}{The standard Python random number generator.}
\end{seealso}


\subsection{The Random Number Generator Interface
            \label{rng-objects}}

% XXX This *must* be updated before a future release!

The \dfn{Random Number Generator} interface describes the methods
which are available for all random number generators.  This will be
enhanced in future releases of Python.

In this release of Python, the modules \refmodule{random},
\refmodule{whrandom}, and instances of the
\class{whrandom.whrandom} class all conform to this interface.


\begin{funcdesc}{choice}{seq}
Chooses a random element from the non-empty sequence \var{seq} and
returns it.
\end{funcdesc}

\begin{funcdesc}{randint}{a, b}
\deprecated{2.0}{Use \function{randrange()} instead.}
Returns a random integer \var{N} such that
\code{\var{a} <= \var{N} <= \var{b}}.
\end{funcdesc}

\begin{funcdesc}{random}{}
Returns the next random floating point number in the range [0.0
... 1.0).
\end{funcdesc}

\begin{funcdesc}{randrange}{\optional{start,} stop\optional{, step}}
Return a randomly selected element from \code{range(\var{start},
\var{stop}, \var{step})}.  This is equivalent to
\code{choice(range(\var{start}, \var{stop}, \var{step}))}.
\versionadded{1.5.2}
\end{funcdesc}

\begin{funcdesc}{uniform}{a, b}
Returns a random real number \var{N} such that
\code{\var{a} <= \var{N} < \var{b}}.
\end{funcdesc}
